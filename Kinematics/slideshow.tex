\documentclass{beamer}
\usepackage{beamerthemeshadow}
\usepackage{xmpmulti}
\usefonttheme[onlymath]{serif}


\graphicspath{ {./graphics/} }



\usecolortheme{rose}
\usetheme{Warsaw}


\title{Petaluma High School}
\author{Mr. Kelley}
\institute{Petaluma High School}
\date{Fall 2012}


\newcommand{\m}{\mbox{ m }}

\newcommand{ \bigLPic}[2]{
       \includegraphics[scale = .34]<#1>{#2}}

\newcommand{ \bigPPic}[2]{
       \includegraphics[angle = -90, scale = .20]<#1>{#2}}

\newcommand{ \halfnhalfLR}[2]{
       \begin{columns}
       \column{.5\textwidth}
       #1
       \column{.5\textwidth}
       #2
       \end{columns}
}
\newcommand{ \halfnhalfBR}[2]{
       \begin{columns}
       \column{.3\textwidth}
       #1
       \column{.7\textwidth}
       #2
       \end{columns}
}

\newcommand{ \halfnhalfBL}[2]{
       \begin{columns}
       \column{.7\textwidth}
       #1
       \column{.3\textwidth}
       #2
       \end{columns}
}

\newcommand{ \halfnhalfRL}[2]{
       \begin{columns}
       \column{.5\textwidth}
       #2
       \column{.5\textwidth}
       #1
       \end{columns}
}


\begin{document}

\section{Kinematics}

\frame{
       \frametitle{Reference Frames}
       
	{\huge If you are floating in empty space, and you see a fellow floater approaching you, how do you know which one of you is moving?}
}

\frame{
       \frametitle{Reference Frames}
       
	{\huge A ``Reference Frame", or zero-point, or origin, \emph{must always be chosen} in order to apply a mathematical model to a physical situation.}
}

\frame{
	\frametitle{Distance vs. Displacement}
	
	Let $x(t)$ be the position of a particle (or ``body") at time $t$. \\
	\pause
	For some body,
	\begin{align*}
	\uncover<3->{x(0) &= 0} \\
	\uncover<4->{x(1) &= 20 \m} \\
	\uncover<5->{x(2) &= 10 \m} 
	\end{align*}
	\uncover<6->{The \emph{displacement} of this body is 10 m, while the total distance it traveled is 30 m.}
     }

\frame{
	\frametitle{Average Velocity vs. Average Speed}
	Same body: \pause
	\begin{align*}
	x(0) &= 0 \\
	x(1) &= 20 \m \\
	x(2) &= 10 \m
	\end{align*}
	\pause
	What is the average velocity and the average speed in the time interval $t=0$ to $t=2$s? \pause
	$$ \mbox{average speed} = \frac{\mbox{distance traveled}}{\mbox{time elapsed}}$$
	$$ \mbox{average velocity} = \frac{\mbox{displacement}}{\mbox{time elapsed}} = \frac{\mbox{final position - initial position}}{\mbox{time elapsed}}$$
	
	}
	
\frame{
	\frametitle{Velocity vs. Speed}
	\begin{center}
	For a new body \pause
	\begin{align*}
	x(0) &= 0 \\
	x(5) &= -20 \m
	\end{align*}
	\pause
	What is the average speed and velocity of this body? \\ \pause
	Velocity is like speed, except direction matters.
	\end{center}
       }
       
\frame{
	\frametitle{Instantaneous Velocity}
	\begin{center}
	Average Velocity:
	$$\overline{v} = \frac{\Delta x}{\Delta t}$$ \pause
	
	\vspace{1cm}
	Instantaneous Velocity:
	$$ v = \lim_{\Delta t \to 0} \frac{\Delta x}{\Delta t} $$ \pause
	$$ v = \frac{dx}{dt} $$
	\end{center}
       }

\frame{
	\frametitle{Average Acceleration}
	Average Acceleration is the change in velocity divided by the change in time:
	$$\overline{a} = \frac{\Delta v}{\Delta t}$$ \pause
	
	\vspace{1cm}
	\begin{center}
	For some body \pause
	\begin{align*}
	v(0) &= 10 \mbox{m/s} \\
	v(5) &= 30 \mbox{m/s} \\
	\end{align*}
	\pause
	What is the average acceleration this body? \\ 
	\end{center}
	}
	
\frame{
	\frametitle{Kinematics Equations}
	\begin{align*}
	v &= v_\circ + at \\ \\
	x &= x_\circ +v_\circ t + \frac{1}{2}at^2 \\ \\
	v^2 &= v_\circ^2 + 2a(x-x_\circ) \\ \\
	\overline{v} &= \frac{v+v_\circ}{2} \\
	\end{align*}
	}

\frame{
	\frametitle{The Graph of $x$ vs. $t$.}
	\begin{center}
	\includegraphics[scale = .8]<1>{xvst5.pdf}
	\includegraphics[scale = .8]<2>{xvst3.pdf} 
	\includegraphics[scale = .8]<3>{xvst4.pdf}  
	\includegraphics[scale = .8]<4>{xvst1.pdf}  
	\end{center}
       }
       
\frame{
	\frametitle{The Graph of $v$ vs. $t$.}
	\begin{center}
	\includegraphics[scale = .8]<1>{vvst1.pdf}
	\includegraphics[scale = .8]<2>{vvst2.pdf} 
	\includegraphics[scale = .8]<3>{vvst3.pdf}  
	\includegraphics[scale = .8]<4>{vvst4.pdf}  
	\end{center}
       }


              
\end{document}
\documentclass[12pt]{report}
\usepackage{amssymb}
\usepackage{amsmath}
\usepackage{multicol}
\usepackage{graphicx}
\usepackage{subfigure}
\usepackage{verbatim}
\usepackage[letterpaper,left=1.5cm,right=2.5cm, top=1.5cm,
bottom=1.9cm,head=0cm,foot=1cm]{geometry}

\parindent=0in

\newcommand{ \probDir}[1]{{ \bf\small #1 \mbox{  }}}

\newcommand{ \breakList}{\setcounter{saveenum}{\value{enumi}} \end{enumerate}}
\newcommand{ \contList}{\begin{enumerate} \setcounter{enumi}{\value{saveenum}}}

\newcommand{ \labTitle}[1]{{ \Large \bf {Lab: #1} \\ {Honors Physics - PHS} \\  {Mr. Kelley} \vspace{.1cm}}}
\newcommand{ \summary}[1]{ \hrulefill \\  {\large \bf Overview \\} \begin{center}\fbox{{\parbox{15cm}{#1}}}\end{center} \vspace{.5cm}}
\newcommand{ \standards}[1]{ \hrulefill \\ {\large \bf Standards \\} #1 }
\newcommand{ \materials}[1]{ \hrulefill \\ {\large \bf Materials \\} #1}
\newcommand{ \swbat}[1]{ \hrulefill \\ {\large \bf Learning Goals \\} #1}
\newcommand{ \assess}[1]{ \hrulefill \\ {\large \bf Assessment \\} #1}
\newcommand{ \accom}[1]{ \hrulefill \\ {\large \bf Accommodations \\} #1}
\newcommand{ \sequence}[1]{ \hrulefill \\ {\large \bf Lesson Sequence \\} \begin{enumerate} #1 \end{enumerate}}
\newcommand{ \seq}[3]{\item {\bf#1.} #3 {\bf(#2)}}


\newcommand{ \intro}[1]{ \hrulefill \\  {\large \bf Introduction \\} \begin{center} \parbox{15cm}{#1} \end{center} \vspace{.5cm}}
\newcommand{ \procedure}[1]{ \hrulefill \\ {\large \bf Procedure \\} \begin{enumerate} #1 \end{enumerate}}


\newcounter{saveenum}

%%%%%%%%%%%%%%%%%%%%%%%%%%%%%%%%%%%%%%%%%
\begin{document}



{\Large \bf Mr. Kelley's Guide to \\ Component Vectors}

\hrule \vspace{.5cm}
{\large \bf The Problem \\ \vspace{.5cm} \\}
Sometimes we have to deal with vectors in 2-dimensions.  We need a reliable way to add them.  Consider the vector $\vec{V}$ that has magnitude 5 and points at an angle of $35^\circ$ with respect to the positive x-axis.


\materials{
\begin{itemize}
\item Plastic Dish (1)
\item Erlenmeyer flask (1) 
\item Candles (1)
\item Matches
\item Water (enough to fill the dish, but not drown the candle)
\item Ice (optional)
\end{itemize}
}

\procedure{
\item Gather materials
\item Fix candle to bottom of dish
\item Fill dish with water, 3-4 cm
\item Light candle
\item Place erlenmeyer flask over burning candle, down to water
\item Observe phenomena
\item Repeat as desired
}
Is there anything about this that you can measure?  What can you alter in order to test theories about what is going on?  Discuss observations with your group members.  Together, create a poster with words, pictures, formulas, or anything else that summarizes your explanation of what you observe.  We will share these as a class.

\unitlength = 1 pt
\begin{picture}(1,1)(-320, -150)

\end{picture}

\end{document}
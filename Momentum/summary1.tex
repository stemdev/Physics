\documentclass[12pt]{report}
\usepackage{amssymb}
\usepackage{amsmath}

\usepackage{multicol}
\usepackage{graphicx}
\usepackage{subfigure}
\usepackage{verbatim}

\usepackage[letterpaper,left=1cm,right=2cm, top=1.5cm,
bottom=1.5cm,head=0cm,foot=1cm]{geometry}

\parindent=0in


\newcommand{\m}{\mbox{ m }}
\newcommand{\kg}{\mbox{ kg }}
\newcommand{\s}{\mbox{ s }}
\newcommand{\ke}{\mbox{\small KE}}
\newcommand{\pe}{\mbox{\small PE}}


\newcommand{ \probDir}[1]{{ \bf\small #1 \mbox{  }}}

\newcommand{ \breakList}{\setcounter{saveenum}{\value{enumi}} \end{enumerate}}
\newcommand{ \contList}{\begin{enumerate} \setcounter{enumi}{\value{saveenum}}}

\newcounter{saveenum}

\def \wspace{5cm}

%%%%%%%%%%%%%%%%%%%%%%%%%%%%%%%%%%%%%%%%%
\begin{document}

{\bf{Honors Physics} \hfill {The Story So Far...} \hfill {Mr. Kelley}} \\ \\
%%%%%%%%%%



\begin{itemize}
\item Kinematics
\begin{itemize}
\item We are often interested in the position $x$ of a particle at a time $t$:   $x(t)$
\item The rate of change of $x(t)$ is velocity, $v(t)$.  In calculus, $x'(t)=v(t)$.
\item The rate of change of $v(t)$ is acceleration, $a(t)$.  Another derivative: $x''(t) = v'(t) = a(t)$
\item The following are mathematical consequences of constant acceleration: \\ \\
$v = \frac{\Delta x}{t}$ \hfill $v_f = v_0 + at$ \hfill $v_f^2 = v_0^2 + 2 a\Delta x$ \hfill $x_f = x_0 + v_0 t + \frac{1}{2}at^2$ \\
\item These equations can be generalized to two dimensions (or 3 or 4, for that matter).  2-dimensional kinematics are used to understand the path of projectiles.
\end{itemize}

\item Conservation of Momentum
\begin{itemize}
\item A mass, $m$, moving at a velocity, $v$, has a momentum defined as $\rho = mv$.
\item Momentum is conserved in the universe in all cases.  We use this law to understand collisions.
\item Collisions can happen in 1 or more dimensions.  For each dimension, $\rho_\circ = \rho_f$.
\end{itemize}

\item Mechanical Energy
\begin{itemize}
\item A mass in motion has energy associated with that motion; Kinetic Energy (\ke)
\item A mass in a gravitational field has an energy associated with its position; \\ Potential Energy (\pe)
\item These two scalar quantities together comprise the mechanical energy of an object. \\
\mbox{} \hfill $\ke=\frac{1}{2}mv^2$ \hspace{2cm} $\pe=mgh$ \hfill \mbox{}
\item Under ideal conditions (conservative forces, no friction) we can use conservation of mechanical energy to understand some things about the motion of a mass.
\item Conservation of mechanical energy:
$$\ke_\circ + \pe_\circ = \ke_f + \pe_f$$
\item Energy is always conserved in the universe, but mechanical energy may not always be conserved.  Energy can be transformed into heat, work, light, or other forms.
\item In more advanced formulations of mechanics (like you would see in advanced undergraduate physics), the motion of a body is determined by minimizing the function $L(x(t), x'(t), x''(t), m) = \ke - \pe$.  In other words, the path a particle will take in nature minimizes the difference of kinetic and potential energies.
\end{itemize}

\end{itemize}


{\bf Homework:} \\ Who was Isaac Newton and what are his three laws?  1 page max \emph{IN YOUR OWN WORDS}.


\end{document}
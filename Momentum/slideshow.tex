\documentclass{beamer}
\usepackage{beamerthemeshadow}
\usepackage{xmpmulti}
\usefonttheme[onlymath]{serif}




\usecolortheme{rose}
\usetheme{Warsaw}


\title{Petaluma High School}
\author{Mr. Kelley}
\institute{Petaluma High School}
\date{Fall 2012}


\newcommand{\m}{\mbox{ m }}

\newcommand{ \bigLPic}[2]{
       \includegraphics[scale = .34]<#1>{#2}}

\newcommand{ \bigPPic}[2]{
       \includegraphics[angle = -90, scale = .20]<#1>{#2}}

\newcommand{ \halfnhalfLR}[2]{
       \begin{columns}
       \column{.5\textwidth}
       #1
       \column{.5\textwidth}
       #2
       \end{columns}
}
\newcommand{ \halfnhalfBR}[2]{
       \begin{columns}
       \column{.3\textwidth}
       #1
       \column{.7\textwidth}
       #2
       \end{columns}
}

\newcommand{ \halfnhalfBL}[2]{
       \begin{columns}
       \column{.7\textwidth}
       #1
       \column{.3\textwidth}
       #2
       \end{columns}
}

\newcommand{ \halfnhalfRL}[2]{
       \begin{columns}
       \column{.5\textwidth}
       #2
       \column{.5\textwidth}
       #1
       \end{columns}
}


\begin{document}

\section{Kinematics}

\frame{
       \frametitle{Momentum and Collisions}
       
	{\huge Q: How can we understand collisions in physics? \pause \\ \vspace{1cm}  A: Conservation of Momentum.}
}

\frame{
       \frametitle{Momentum and Collisions}
       
       \begin{center}
	The momentum of a particle of mass $m$ and velocity $v$ is defined as $$ \vec{\rho} = m\vec{v}$$ \pause
	That is a ``rho", not a p! \pause \\
	Momentum is a \emph{vector} quantity!
	
	\end{center}
}

\frame{
	\frametitle{Momentum and Collisions}
	
	What is the momentum of a 1,052 kg Honda Civic traveling at 100 km/h?
     }

\frame{
	\frametitle{Momentum and Collisions}
	
	{\centering \huge The Law of Conservation of Momentum} \\ \pause
	In a \emph{closed system}, total momentum remains constant. \\ \pause
	Therefore, when comparing two different moments in a closed system: \pause
	
	$$\mbox{total initial momentum} = \mbox{total final momentum}$$
     }


\frame{
	\frametitle{Momentum and Collisions}
	
	Now the 1,052 kg Honda traveling at 100 km/h crashes into a stationary pickup.  The two vehicles stick together and roll forward.  How fast is the mess rolling forward? \\ \pause The pickup has a mass of 2,000 kg.
     }

\frame{
	\frametitle{Momentum and Collisions}
	
	What is the recoil velocity of a 5.0 kg rifle that shoots a 0.020 kg bullet at 620 m/s?
     }


\frame{
	\frametitle{Momentum and Collisions}
	
	A billiard ball of mass $m_1=0.2$ kg and $v_1 = 10 $m/s, $0^\circ$ collides with another ball of the same mass, initially at rest.  The first ball is deflected by $30^\circ$ and has a final velocity of 8 m/s. \pause \\ What is the final velocity of the second ball?
     }
	
\frame{
	\frametitle{Center of Mass}
	I've been goofing you, here's the full story: \pause \\ \vspace{2cm}
	
	\emph{A large, extended body of mass $m$ can be physically/mathematically modeled as a point-particle located at the body's center of mass.}
	\\ \vspace{1cm} \pause
	Alternately stated: \\ \vspace{1cm} \emph{The center of mass of a system moves as if the mass were concentrated at that point}
       }
       
\frame{
	\frametitle{Center of Mass}
	
	$$ x_{\mbox{\tiny CM}} = \frac{m_1 x_1 + m_2 x_2 + \dots + m_n x_n}{m_1+m_2+\dots+m_n}$$
	\vspace{1cm} \pause
	$$ x_{\mbox{\tiny CM}} = \frac{\sum\limits_{i=1}^{n} m_i x_i}{\sum\limits_{i=1}^{n} m_i}$$
       }

\frame{
	\frametitle{Center of Mass}
	
	Three point masses $a,b,$ and $c$ are located at (1,4), (-2,2), and (3,-1), respectively.
	$$m_a=3\mbox{g}, m_b=4\mbox{g}, m_c=5\mbox{g}$$
	
	Where is the center of mass located?
       }

              
\end{document}
\documentclass{beamer}
\usepackage{beamerthemeshadow}
\usepackage{xmpmulti}
\usefonttheme[onlymath]{serif}




\usecolortheme{rose}
\usetheme{Warsaw}


\title{Petaluma High School}
\author{Mr. Kelley}
\institute{Petaluma High School}
\date{Fall 2012}


\newcommand{\m}{\mbox{ m }}
\newcommand{\kg}{\mbox{ kg }}
\newcommand{\s}{\mbox{ s }}
\newcommand{\ke}{\mbox{\small KE}}
\newcommand{\pe}{\mbox{\small PE}}


\newcommand{ \bigLPic}[2]{
       \includegraphics[scale = .34]<#1>{#2}}

\newcommand{ \bigPPic}[2]{
       \includegraphics[angle = -90, scale = .20]<#1>{#2}}

\newcommand{ \halfnhalfLR}[2]{
       \begin{columns}
       \column{.5\textwidth}
       #1
       \column{.5\textwidth}
       #2
       \end{columns}
}
\newcommand{ \halfnhalfBR}[2]{
       \begin{columns}
       \column{.3\textwidth}
       #1
       \column{.7\textwidth}
       #2
       \end{columns}
}

\newcommand{ \halfnhalfBL}[2]{
       \begin{columns}
       \column{.7\textwidth}
       #1
       \column{.3\textwidth}
       #2
       \end{columns}
}

\newcommand{ \halfnhalfRL}[2]{
       \begin{columns}
       \column{.5\textwidth}
       #2
       \column{.5\textwidth}
       #1
       \end{columns}
}


\begin{document}

\section{Energy}

\frame{
       \frametitle{Mechanical Energy}
       
	{\huge Q: What is Energy?? \pause \\ \vspace{1cm}  A: Energy takes many forms, but \emph{Mechanical Energy} is the energy due to the motion and position of a mass.}
}

\frame{
	\frametitle{Mechanical Energy}
	
	\begin{center}
	There are two kinds of Mechanical Energy; \emph{Kinetic Energy} and \emph{Potential Energy}.
	\end{center}
	$$\ke = \frac{1}{2}mv^2$$
	$$\pe = mgh$$
	(Note: the above equation for \pe \mbox{ }is for \emph{gravitational potential energy}.  There are other kinds of potential energy.  For example; \pe which is stored in a spring.)
}

\frame{
	\frametitle{Mechanical Energy}
	
	Sputnik 1 (1957), the first ever satellite, circled the globe at 29,000 km/hr at a height of 577 km. \pause \\
	What is the kinetic and potential energy of the satellite? \pause \\
	The mass of Sputnik was 84 kg.  \pause Assume the gravitational force is the same as it is at the surface of the earth.
	
     }

\frame{
	\frametitle{Mechanical Energy}
	
	Notice that mechanical energy (\ke \mbox{ }and \pe) are \emph{scalar quantities}.  \pause \\ {\large \bf WE ARE NOT CONCERNED WITH DIRECTION HERE, \\ \pause only \emph{POSITION} and \emph{SPEED}.}
     }


\frame{
	\frametitle{Mechanical Energy}
	{\huge Energy is conserved in the universe.} \\ \pause \vspace{.5cm}
	Therefore:
	$$\ke_\circ +\pe_\circ = \ke_f + \pe_f$$
	
     }

\frame{
	\frametitle{Mechanical Energy}
	
	Suppose someone drops a 3kg bowling ball, initially at rest, off a 100m high cliff.  How fast will it be going when it hits the bottom?
	
     }

\frame{
	\frametitle{Units}
	
	We see that mechanical energy has units of $\kg \cdot \m^2 / \s^2$.  We give this a name, "Joule."
	
	$$\frac{1\kg \cdot \m^2}{ \s^2} = 1 \mbox{J}$$
	
     }

\end{document}